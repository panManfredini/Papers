% ****** Start of file apssamp.tex ******
%
%   This file is part of the APS files in the REVTeX 4.1 distribution.
%   Version 4.1r of REVTeX, August 2010
%
%   Copyright (c) 2009, 2010 The American Physical Society.
%
%   See the REVTeX 4 README file for restrictions and more information.
%
% TeX'ing this file requires that you have AMS-LaTeX 2.0 installed
% as well as the rest of the prerequisites for REVTeX 4.1
%
% See the REVTeX 4 README file
% It also requires running BibTeX. The commands are as follows:
%
%  1)  latex apssamp.tex
%  2)  bibtex apssamp
%  3)  latex apssamp.tex
%  4)  latex apssamp.tex
%
\documentclass[%
 reprint,
%superscriptaddress,
%groupedaddress,
%unsortedaddress,
%runinaddress,
%frontmatterverbose, 
%preprint,
%showpacs,preprintnumbers,
%nofootinbib,
%nobibnotes,
%bibnotes,
 amsmath,amssymb,
 aps,
%pra,
%prb,
%rmp,
%prstab,
%prstper,
%floatfix,
]{revtex4-1}

\usepackage{graphicx}% Include figure files
\usepackage{dcolumn}% Align table columns on decimal point
\usepackage{bm}% bold math
%\usepackage{hyperref}% add hypertext capabilities
%\usepackage[mathlines]{lineno}% Enable numbering of text and display math
%\linenumbers\relax % Commence numbering lines

%\usepackage[showframe,%Uncomment any one of the following lines to test 
%%scale=0.7, marginratio={1:1, 2:3}, ignoreall,% default settings
%%text={7in,10in},centering,
%%margin=1.5in,
%%total={6.5in,8.75in}, top=1.2in, left=0.9in, includefoot,
%%height=10in,a5paper,hmargin={3cm,0.8in},
%]{geometry}

\begin{document}

%\preprint{APS/123-QED}

\title{Search for WIMP Inelastic Scattering Off Xenon Nuclei With Xenon100 Data}% Force line breaks with \\
\thanks{}%

\author{Ann Author}
 \altaffiliation[Also at ]{Physics Department, XYZ University.}%Lines break automatically or can be forced with \\
\author{Second Author}%
 \email{Second.Author@institution.edu}
\affiliation{%
 Authors' institution and/or address\\
 This line break forced with \textbackslash\textbackslash
}%

\collaboration{XENON Collaboration}%\noaffiliation

\author{Authors}
% \homepage{http://www.Second.institution.edu/~Charlie.Author}
\affiliation{
 Second institution and/or address\\
 This line break forced% with \\
}%
\affiliation{
 Affiliation
}%

%\collaboration{CLEO Collaboration}%\noaffiliation

\date{\today}% It is always \today, today,
             %  but any date may be explicitly specified

\begin{abstract}

\end{abstract}

%\pacs{Valid PACS appear here}% PACS, the Physics and Astronomy
                             % Classification Scheme.
%\keywords{Suggested keywords}%Use showkeys class option if keyword
                              %display desired
\maketitle

%\tableofcontents

\section{\label{sec:level1} Introduction}
Astrophysical evidence indicates that the dominant mass fraction of our Universe consists of some yet unknown form
of dark matter. Well motivated models predict Dark Matter in the form of Weakly Interacting Massive Particles (WIMPs),
hypothesis which is currently being tested by several direct and indirect detection experiment.

Most of direct detection searches focuses on  elastic scattering of dark matter particles off nuclei.
In this analysis instead we explore an inelastic scattering process, we consider the $^{129}\text{Xe}$ isotope being excited to a low-lying state
with subsequent prompt de-excitation via the emission a photon. This isotope is an excellent target
since its abundance in natural xenon is of 26.4\% and a relatively low energy is necessary to excite its $3/2+$ state above the $1/2+$
spin ground state.
Inelastic WIMP-nucleus scattering in xenon is complementary to elastic scattering for spin-dependent interactions,
the former dominates the integrated rate above  $\simeq10$~keV of energy deposition.
Furthermore, in the case of dark matter detection, this channel can be employed to asses whether the nature of the fundamental interaction
is spin-dependent or not.

\section{Xenon100 Detector}
The Xenon100 experiment is a  dual phase liquid xenon TPC. For a given interaction in the liquid target this type of detector produces two separated signals,
one proportional to the prompt scintillation (S1) the other to ionization (S2).

To add: sentences about detector stability, science run data used, Ly and Y measurements used. 

\section{Data Analysis}

brief explaination of the signature, selection cuts, few words about acceptances, image of signal region and control region.

\subsection{Signal Simulation} 

description of the simulated signal, few words about cross checks MC matching.

\subsection {Background Model}
Description of the data driven bkg model evaluation, few numbers on estimated background, words about cross checks with Th232.

%A citation in text uses the command \verb+\cite{#1}+ or
%\verb+\onlinecite{#1}+ and refers to an entry in the bibliography. 
%An entry in the bibliography is a reference to another document.
\subsection{Systematic Uncertainties}

few words, mainly a table summarizing uncertainties.

\section{Results}
No evidence of dark matter.

\subsubsection{Citations}
%Because REV\TeX\ uses the \verb+natbib+ package of Patrick Daly, 
%the entire repertoire of commands in that package are available for your document;
%see the \verb+natbib+ documentation for further details. Please note that
%REV\TeX\ requires version 8.31a or later of \verb+natbib+.


\end{document}
%
% ****** End of file apssamp.tex ******

