\section{Data Analysis}

This analysis is performed using XENON100 Run-II science data, which corresponds to an exposure of 224.6~live days. The detector response to
electronic recoil (ER) has been characterized using $^{60}$Co and $^{232}$Th radioactive sources, while response to inelastic nuclear recoil (NR)
scattering was calibrated using an $^{241}$AmBe source. 

The inelastic scattering of a WIMP with the nucleus of $^{129}$Xe produces an energy deposit via nuclear recoil with subsequent emission of  
a 39.6~KeV de-exitation photon. 
The largest fraction of the energy released in the event is via electronic recoil due to the emitted photon, this represents an
unusual signature for this kind of detector and brings the possible signal to overlay a phase space region with large backgrounds.
The choosen region of interest for this analysis surrounds the 39.6~KeV xenon line in the cS1-cS2 plane which is further divided into
sub-regions, as shown in Figure~\ref{fig:SR}.

Events are asked, other than falling in the defined region of interest, to fullfill several selection criteria which can be sumarized as:
selection aimed to reduce noise impact including energy (S1) and S2 thresholds, events must be of single scatter nature and fall 
into a predefined fiducial volume. This analysis follows the selection criteria described in detail in \cite{dataAnalysis} for Run-II, 
with only few exceptions. 
%only few modification have been designed specifically for this analysis and discussed below. 
In particular, the selection on S2 width as a function of drift time has been optimized on a sample of events selected from the 39.6~KeV line
and set to a 95\% acceptance on these. Events are required to be single scatter by applying a threshold on the 
second largest S2 peak size,  for this analysis this threshold has been optimized to 160~PE and set constant as function of S2 signal size. 
Finally the chosen fiducial volume corresponds to 34~Kg of liquid xenon.


\begin{figure}[t!]
  \includegraphics[width=\linewidth]{images/bkg_in_sr.png}
  \caption{Signal region and control region.}
  \label{fig:SR}
\end{figure}



\subsection{Signal Simulation} 
%description of the simulated signal, few words about cross checks MC matching.
The detector response to inelastic scattering of WIMPs  off $^{129}$Xe nucleus was simulated using an empirical model.
The total deposited energy is divided into two independent contributions: the one related to the 39.6~KeV de-exitation photon and the one  relative to
the simultaneous nuclear recoil of the xenon atom, the number of photons and charge yield detected is simulated separatedly for each contribution
and then added togheter.

The ER induced by the de-exitation photon is simulated assuming a two dimensional normal distribution as its pdf in the cS1-cS2 plane, $f(cS1,cS2)$, 
described (except a normalization factor) in equation~\ref{f:2dgaus}:
\begin{multline}
	f(cS1,cS2)  = exp \Big( -\frac{1}{2(1-\rho^2)} \Big[ \frac{(cS1 - \mu_{cS1})^2}{\sigma_{cs1}^2} + \\ 
	 \frac{(cS2 - \mu_{cS2})^2}{\sigma_{cs2}^2} - \frac{2\rho(cS1 - \mu_{cs1}) (cS2 - \mu_{cs2})} {\sigma_{cs1}\sigma_{cs2}} \Big] \Big) 
	%f(cS1,cS2) ~ = ~  \frac{1}{2 \pi \sigma_{cs1} \sigma_cs2 \sqrt{1-\rho^{2}} } exp \Big( -\frac{1}{2(1-\rho^2)} \Big[ \frac{(cS1 - \hat{cS1})^2}{\sigma_{cs1}^2} \Big] \Big)
\label{f:2dgaus}
\end{multline}
where $\mu_{cS1}$ and $\mu_{cS2}$ 
represents the average observed cS1 and cS2 given a 39.6~KeV ER, $\sigma_{cs1}$ and $\sigma_{cs2}$ are the standard deviation in cS1 and cS2 respectively,
while $\rho$ stands for the correlation between cS1 and cS2.
The detector related light yield $L_y$ measured at 39.6~KeV, necessary to evaluate the average number of photon detected ($\mu_{cS1}$),
%with AmBe calibration has been found to have a large systematic uncertainty
%due to the contribution of the nuclear recoil, so the light yield used to describe the number of photon detected 
is obtained as the result of a NEST model~\cite{NEST,Geant1,Geant2} fit to data collected with several lines.  The same model is used to predict the charge yield at
39.6~KeV which is then scaled according to the detector's secondary scintillation gain $Y$, determined from detector response to single electrons~\cite{SingleE}.
Note that the corrected S2 observed by the bottom PMT array is used in this analysis.  
The detector resolution at 39.6~KeV in cS1 and cS2 has been measured to be respectively 15.8\% and 14.7\% and used to extract the standard 
deviations $\sigma_{cs1}$, $\sigma_{cs2}$.  The correlation parameter is assumed to be independent of energy (at least in the considered range) and measured
using the 164~KeV xenon activated line by $^{124}$AmBe calibration data, this line is choosen since allows to disentangle efficiently contribution from nuclear recoil.
The measured correlation is $\rho \, = \, -0.45 \pm 0.10$. 


The cS1 and cS2 distributions from NR contribution are predicted starting from the nuclear recoil energy spectrum
for WIMPs inelastic interaction~\cite{inelastic_th}. The average cS1 and cS2 are given by equations~\ref{f:cs1} and~\ref{f:cs2} respectively,
where $\mathcal{L}_{eff}$ is the liquid xenon NR relative scintillation efficiency while $S_{ee} \, = \, 0.58$  and $S_{nr} \, = \, 0.95$ describe the scintillation 
quenching due to the electric field \cite{ScintQuenching}. The parameterization and uncertainties of $\mathcal{L}_{eff}$ as a function of $E_{nr}$ are based on existing 
direct measurements \cite{run8Result}. The light yield at 122~keVee originate from the same NEST model fit as described above. For the cS2 the parameterization 
of $Q_{Y}(E_{nr})$ is taken from \cite{QY}.
Finally all detector related resolution effects are introduced following the prescriptions described in \cite{dataAnalysis}.

\begin{equation}
cS1_{nr} ~=~ E_{nr} \, \mathcal{L}_{eff}(E_{nr}) \, L_{y} \, \frac{S_{nr}}{S_{ee}}
\label{f:cs1}
\end{equation}

\begin{equation}
cS2_{nr}  ~ = ~ E_{nr} \, Q_{Y}(E_{nr}) \, Y
\label{f:cs2}
\end{equation}

The pdf of the ER and NR contributions are then convoluted toghether to obtain the overall pdf of the signal.
%The two contribution are summed toghether in an event-by-event basis using random extraction from the two pdfs. 
A 2D (cS1 versus cS2) acceptance map is applied to the signal pdf to reproduce data selection effects. Acceptances are computed separately for each selection 
criteria using $^{124}$AmBe calibration sample, selections as the outer volume veto and the single scatter interaction represent an exception  and 
a detailed computation has been performed in these cases. The acceptance average in the region of interest to about $0.80 \pm 0.05$. 
Figure~\ref{fig:SR} shows an example of full simulated signal model for a WIMP of 100~GeV mass. 

The signal simulation procedure has been validated reproducing the 39.6~KeV xenon line from interaction due to
$^{124}$AmBe source and has been compared to data. For the comparison the proper  $^{124}$AmBe nuclear recoil and acceptances
were simulated. The simulated events were found in agreement  with calibration data within statistical uncertainties.

\begin{figure}[t!]
  \includegraphics[width=\linewidth]{images/wimp_in_sr.png}
  \caption{Signal region and control region, for WIMP of mass 100~GeV.}
  \label{fig:SR2}
\end{figure}


\subsection {Background Model}

The main expected background contribution in the region of interest is due material radioactivity, composed mainly by photons  
interacting via compton scattering. Background contribution due to the activation of the xenon 39.6~KeV line from radiogenic neutrons is expected to be negligible.
The background is modeled using data from $^{60}$Co calibration campaign, which are assumed to well represent the background density distribution 
in the cS1-cS2 plane. 

The calibration sample yields  about 22'000 events in the ROI, these are then scaled to data according to a measured background scale factor $\tau_{bkg}$.
The scale factor is measured in the two control regions shown in Figure~\ref{fig:SR} and labelled CR1 and CR2. The two control 
regions give compatible results and the computed average is $\tau_{bkg} \, =  \, 0.034 \pm 0.002 $, where the reported uncertainty 
is of statistical nature only.

The distribution of the calibration sample has been compared to the data of the sience run in the two control regions,
agreement is found within statistical uncertainties. $^{60}$Co calibration data have been compared in the region of interest to  
data from $^{232}$Th calibration campaign, the largest deviation between the two shapes is within 4\%. An additional systematic
uncertainty of 4\% has been applied to the expected background yield of each sub-region of the ROI.




\subsection{Systematic Uncertainties}

%FIXME: maybe change region definitions in order to have the sub-region 9 labelled as 7, so the signal would "nicely" look like a bump.
\begin{figure}[t!]
  \includegraphics[width=\linewidth]{images/wimp_sys_unc.png}
  \caption{Signal region, uncertainties for WIMP of mass 100~GeV.}
  \label{fig:unc}
\end{figure}


Uncertainties on the total prediction of background events arise from the uncertainty on the measure of the normalization 
factor, $\tau_{bkg}$, and amount to 6\%, contribution of radiogenic neutrons are neglected. 
Systematic uncretainty on the shape of the predicted bakground distribution are assesed by the maximal discrepancy in the ROI between
the $^{60}$Co and $^{232}$Th calibration samples, a 4\% systematic additional to statistical uncertainty is assigned to the expected yield of each sub-region,
note that uncertainties belonging to different sub-regions are considered independent from each other.
%It is important not only to describe correctly the expected total background but also its distribution

Signal model, Energy scale uncertainties.... Uncertainties related to energy scale and more generally to detector response 
are taken into account by relative uncertainties on the measure of $L_y$,$\mathcal{L}_{eff}$,$Y$, $Q_Y$ and $\rho$. Has been shown 
by the simulation that these type of uncertainty affect the total signal yield in the ROI by less than a percent for a wide range of masses,
however they greatly affects the pdf of the signal model in the ROI. For each WIMP mass several signal simulation sample are produced 
warying these parameter respectively $\pm 1 \sigma$, for each region is computed an uncertainty by adding in quadrature the variations
%FIXME: explain this better, the up variation are separated from down variations and added togheter
with respect to nominal of each case, Figure~\ref{fig:unc} is an example of such a systematic uncertainty computation for WIMP mass of 100~GeV.

Uncertainty on the total yield of the signal arising from uncertainties on the selection acceptance are found to be very weakly dependent on 
the WIMP mass, an overall 6\% acceptance uncertainty is then applied to all WIMP hypotesis.

%FIXME: express this in an understandable way
All the uncertainties discussed here are threated as nuissance parameter in the likelihood, which are contrained by a gaussian distribution.


















