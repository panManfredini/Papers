
\appendix*

\section{}

\begin{figure*}[h!]
	\subfigure[]{\includegraphics[width=0.49\linewidth]{images/mc_ambe_comp_contour.png}}
	\subfigure[]{\includegraphics[width=0.49\linewidth]{images/mc_ambe_comp_slice1.png}} \\
	\subfigure[]{\includegraphics[width=0.49\linewidth]{images/mc_ambe_comp_slice2.png}}
	\subfigure[]{\includegraphics[width=0.49\linewidth]{images/mc_ambe_comp_slice3.png}}
	\caption{A simulation of the 39.6\,keV  xenon line activated from $^{243}$AmBe neutrons is compared with  data.
		 Figure~(a) compares contours of equal density in the (cS1,cS2)-plane, while Figures~(b), (c) and (d) show the same distribution projected in cS1 for
		 several ranges of cS2, the histograms are normalized to unit area. Light and dark blue represent simulation and data, respectively.
		}
		
  \label{fig:mc_comp}
\end{figure*}

The detector response to inelastic WIMP-$^{129}$Xe interactions was simulated using an empirical signal model. 
The procedure, described in detail in section~\ref{sec:signal}, takes advantage of several 
approximations that have been validated extensively. 
The main aim of the cross check was to reproduce the 39.6\,keV xenon line activated from $^{124}$AmBe neutrons with simulation.
For this purpose the NR energy spectrum expected from inelastic neutron-$^{129}$Xe scattering has been obtained via  Monte Carlo techniques, 
where we take into account the  detector response and the non-uniform spatial distribution. The Acceptance of analysis selections to this type of interaction 
have been recomputed, in particular, the acceptance to the double scatter cut differs greatly between neutrons and WIMPs scattering. 
Except for acceptances and NR energy spectrum the simulation has been performed following the recipe described in the main text. Figure~\ref{fig:mc_comp}
shows a comparisons between simulation (light blue) and calibration data (dark blue), contour lines of equal densities are compared in Figure~(a), 
while Figures~(b), (c) and (d) show the cS1 projected distributions for different ranges in cS2.


