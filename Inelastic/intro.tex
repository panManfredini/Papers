\section{\label{sec:level1} Introduction}
Astrophysical evidence indicates that the dominant mass fraction of our Universe consists of some yet unknown form
of dark matter. Well motivated models predict Dark Matter in the form of Weakly Interacting Massive Particles (WIMPs),
hypothesis which is currently being tested by several direct and indirect detection experiment. 
\textcolor{blue}{some citation missing.}

Most of direct detection searches focuses on  elastic scattering of dark matter particles off nuclei.
In this analysis instead an inelastic scattering process is explored: we consider the $^{129}\text{Xe}$ isotope being excited to a low-lying state
with subsequent prompt de-excitation via the emission of a photon. This isotope is an excellent target
since its abundance in natural xenon is of 26.4\% and a relatively low energy is necessary to excite its $3/2+$ state above the $1/2+$
spin ground state. These type of processes were previously studied in detail for liquid xenon detectors in~\cite{inelastic_th}.
Inelastic WIMP-nucleus scattering in xenon is complementary to elastic scattering for spin-dependent interactions,
the former dominating the integrated rate above  $\simeq10$~keV of energy deposition.
Furthermore, in case of dark matter detection, this channel can be employed to asses whether the nature of the fundamental interaction
is spin-dependent or not.


\section{Xenon100 Detector}
The Xenon100 experiment is a  dual phase liquid xenon TPC. For a given interaction in the liquid target this type of detector produces two separated signals,
one proportional to the prompt scintillation (S1) the other to ionization (S2).

\textcolor{blue}{To add: describe and explain cS1 and cS2 definitions, some sentences about detector stability, maybe science run data used and calibration campaign goes here, 
maybe Ly and Y measurements used goes here.} 
Note that the corrected S2 observed by the bottom PMT array is used in this analysis.  

