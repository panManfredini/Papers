\section{\label{sec:intro} Introduction}

Astrophysical and cosmological evidence indicates that the dominant mass fraction of our Universe consists of some yet unknown
form of dark, or invisible matter. The dark matter could be made of stable or long-lived and yet undiscovered particles. Well-motivated
theoretical models beyond the Standard Model of particle physics predict the existence of Weakly Interacting Massive
Particles (WIMPs), which are natural candidates for dark matter. This hypothesis is currently being tested by several direct
and indirect detection experiments, as well as at the LHC~\cite{Bertone:2010zza,Baudis:2016qwx}.

Most direct detection searches focus on elastic scattering of galactic dark matter particles off nuclei, where the keV-scale 
nuclear recoil energy is to be detected~\cite{Undagoitia:2015gya,Baudis:2015mpa}. In this work, the 
alternative process of inelastic scattering is explored, where a WIMP-nucleus scattering induces a transition to a low-lying 
excited nuclear state. The experimental signature is a nuclear recoil detected together with the prompt de-excitation 
photon~\cite{Ellis:1988nb}. 

We consider the $^{129}\text{Xe}$ isotope, which has an abundance of 26.4\% in natural xenon, and a lowest-lying 
3/2$^{+}$ state at 36.6\,keV above the 1/2$^+$ ground state. The electromagnetic nuclear decay has a half-life of 0.97\,ns. 
The signatures and structure functions for inelastic scattering in xenon have been studied in detail in~\cite{Baudis:2013bba}. It was found that this channel is complementary to spin-dependent, elastic scattering, and that it dominates the integrated rates above $\simeq10$\,keV of deposited energy. 
In addition, in case of a positive signal, the observation of inelastic scattering would provide a clear 
indication of the spin-dependent nature of the fundamental interaction. 

This paper is structured as follows.  In Section~\ref{sec:xenon100} we briefly describe the main features of the XENON100 detector.  In Section~\ref{sec:analysis} we introduce the data 
sets employed in this analysis and detail the data analysis method, including the simulation of the expected signal and the 
background model. We conclude in Section~\ref{sec:results} with our results, and discuss the new constraints on inelastic WIMP-nucleus interactions.

\section{The XENON100 Detector}
\label{sec:xenon100}

The XENON100 experiment operates a dual-phase (liquid and gas) xenon time projection chamber (TPC) at the Laboratori Nazionali 
del Gran Sasso (LNGS) in Italy. It contains 161\,kg of xenon in total,  with  62\,kg in the active region of the TPC. These 
are monitored by 178 1-inch square, low-radioactivity, UV-sensitive photo-multiplier tubes (PMTs) arranged in two arrays, one in the liquid 
and one in the gas. The PMTs detect the prompt scintillation (S1) and the delayed, proportional scintillation signal (S2) 
created by a particle interacting in the active TPC region. The S2-signal is generated due to ionization electrons, drifted 
in an electric field of 530\,V/cm and extracted into the gas phase by a stronger field of $\sim$ 12\,kV/cm, where the proportional scintillation, or electroluminiscence, 
takes place.
The horizontal position, $(x,y)$, of the interaction site is reconstructed from the position of the S2 shower, while the depth of the interaction, $z$, is given by the drift time measurement.
%These photons carry the $(x,y)$ information of the interaction site, while the $z-$information comes from the drift time measurement. 
The TPC thus yields a three-dimensional event localization, with an $(x,y)$ resolution of $<$3\,mm (1\,$\sigma$), and a $z$ resolution of  $<$0.3\,mm (1\,$\sigma$), enabling to reject the majority of background events via fiducial volume selections~\cite{Aprile:2011dd}. The ratio S2/S1 provides the basis for distinguishing between nuclear recoils (NRs), as induced by fast neutrons and expected from elastic WIMP-nucleus scatters, and electronic recoils (ERs) produced by $\beta$- and $\gamma$-rays.  A 4\,cm thick liquid xenon (LXe) layer surrounds the TPC and is monitored by 64 1-inch square PMTs, providing an effective active veto for further background reduction.

XENON100 has acquired science data between 2008-2015, and has set competitive constraints on spin-independent~\cite{Aprile:2012nq,Aprile:2016swn} 
and spin-dependent~\cite{Aprile:2013doa,Aprile:2016swn} elastic WIMP-nucleus 
scatters, on solar axions and galactic ALPs~\cite{Aprile:2014eoa}, as well as on leptophilic dark matter models~\cite{Aprile:2015ade,Aprile:2015ibr,Aprile:2017yea}. 
%%%% possible to be CUTTED %%%%%%%%% 
%that were invented to explain the annual modulation signature observed in the DAMA/LIBRA experiment~\cite{Bernabei:2012in}.

%%%%%% possible to be CUTTED %%%%%%%%%
Here we explore a new potential dark matter interaction channel in the XENON100 detector, caused by spin-dependent, inelastic WIMP-$^{129}$Xe interactions. The expected inelastic scattering signature is a combination between an ER and a NR, due to the short lifetime of the excited nuclear state and  the short mean free path of $\sim$0.15\,mm of the 39.6\,keV de-excitation photon. 


