\section{\label{sec:level1} Introduction}
Astrophysical evidence indicates that the dominant mass fraction of our Universe consists of some yet unknown form
of dark matter. Well motivated models predict Dark Matter in the form of Weakly Interacting Massive Particles (WIMPs),
hypothesis which is currently being tested by several direct and indirect detection experiment.

Most of direct detection searches focuses on  elastic scattering of dark matter particles off nuclei.
In this analysis instead we explore an inelastic scattering process, we consider the $^{129}\text{Xe}$ isotope being excited to a low-lying state
with subsequent prompt de-excitation via the emission a photon. This isotope is an excellent target
since its abundance in natural xenon is of 26.4\% and a relatively low energy is necessary to excite its $3/2+$ state above the $1/2+$
spin ground state.
Inelastic WIMP-nucleus scattering in xenon is complementary to elastic scattering for spin-dependent interactions,
the former dominates the integrated rate above  $\simeq10$~keV of energy deposition.
Furthermore, in the case of dark matter detection, this channel can be employed to asses whether the nature of the fundamental interaction
is spin-dependent or not.

\section{Xenon100 Detector}
The Xenon100 experiment is a  dual phase liquid xenon TPC. For a given interaction in the liquid target this type of detector produces two separated signals,
one proportional to the prompt scintillation (S1) the other to ionization (S2).

To add: sentences about detector stability, science run data used, Ly and Y measurements used. 

